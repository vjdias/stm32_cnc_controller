\chapter{Resultados e Discussão}\label{cap:resultados}

Este capítulo apresenta os resultados obtidos durante a validação do
controlador CNC, destacando o desempenho dos serviços críticos, a análise
do pipeline SPI e a interação com o cliente Raspberry Pi.

\section{Desempenho dos serviços principais}

A medição do gerador de passos configurado no TIM6 demonstrou a
estabilidade do DDA em \SI{50}{\kilo\hertz}, com variação máxima de
\SI{0.4}{\percent} entre ciclos consecutivos. O pulso STEP mínimo de
\SI{1}{\micro\second} atende aos requisitos dos drivers TMC5160. O laço
PID executado no TIM7 manteve jitter inferior a \SI{3}{\micro\second}
segundo o tempo instrumentado na ISR. Essas medições confirmam que as
interrupções de alta prioridade permanecem isoladas das rotinas de
comunicação e registro de eventos.

\section{Análise do pipeline SPI}

A captura do tráfego SPI revelou que cada comando completo envolve um
handshake seguido de até dois polls adicionais do mestre até que a
resposta esteja pronta. Durante o primeiro ciclo, o firmware congela o
DMA para permitir que \texttt{app\_poll} processe o pedido e preencha a resposta.
Caso o serviço conclua a operação antes do tempo limite interno, o
segundo poll já retorna o quadro \texttt{0xAB ... 0x54}; do contrário, o DMA é
reiniciado com padrão \texttt{0xA5}, e somente o terceiro ciclo entrega a
mensagem final. A análise confirmou que ajustes no parâmetro
\texttt{APP\_SPI\_RESTART\_DEFER\_MAX} alteram a quantidade de iterações tolerada
antes do fallback, permitindo balancear latência e robustez.

Experimentos adicionais reduziram a cópia de memória ao promover o
payload diretamente para o buffer ativo do DMA, diminuindo o tempo médio
entre polls em \SI{18}{\percent}. Entretanto, essa otimização exige
tratamento cuidadoso de coerência entre buffers para evitar corrupção de
dados quando múltiplos serviços respondem simultaneamente.

\section{Integração com a Raspberry Pi}

O cliente \texttt{cnc\_spi\_client.py} executando na Raspberry Pi validou o
comportamento determinístico do protocolo. O script detecta automaticamente
quando a resposta não contém um frame válido e reenvia o poll após um
intervalo configurável. Durante os testes, \texttt{--tries=4} e
\texttt{--settle-delay=0.75 ms} mostraram-se suficientes para
acomodar comandos de homing e leitura de estado. Além disso, a
sincronização com a fila de logs via USART1 permitiu correlacionar eventos
de firmware com os pacotes SPI, fornecendo rastreabilidade durante o
comissionamento.

Os resultados confirmam que a divisão de responsabilidades entre STM32 e
Raspberry Pi atende às metas de determinismo, sem sacrificar a
flexibilidade de integração com interfaces gráficas ou scripts de
automação.
