\chapter{Metodologia}\label{cap:metodologia}

A metodologia adotada para o desenvolvimento do controlador CNC seguiu
um roteiro incremental que prioriza a validação dos blocos críticos antes
de incorporar funcionalidades auxiliares. Cada etapa foi documentada,
testada e revisada de forma a garantir que os requisitos de determinismo
fossem mantidos durante todo o processo.

\section{Roteiro incremental de bring-up}

O ponto de partida consistiu na configuração do clock principal para
\SI{80}{\mega\hertz} e na definição das prioridades do NVIC de acordo
com o orçamento temporal: interrupções externas de segurança, TIM6,
SPI1/DMA, TIM7 e USART1. Em seguida, foram ativadas as entradas de parada
de emergência (E-STOP) e sensores de proximidade, assegurando que flags
de segurança pudessem interromper o fluxo de comandos. As etapas
posteriores focaram na calibração dos temporizadores: o TIM6 foi
dimensionado com $PSC = 79$ e $ARR = 19$, enquanto o TIM7 recebeu
$PSC = 7999$ e $ARR = 9$. Os temporizadores TIM2, TIM3 e TIM5 foram
configurados em modo quadratura para leitura de encoders.

A configuração do SPI1 em modo escravo com DMA circular foi realizada
após os temporizadores, evitando contenda na memória compartilhada.
Por fim, a USART1 foi ajustada para \SI{115200}{bps} e integrada a um
serviço de log com fila não bloqueante. Cada etapa do roteiro foi
acompanhada por testes de bancada: medições de frequência com
osciloscópio, leitura de contadores de encoder e injeção de quadros SPI
utilizando o cliente Python.

\section{Arquitetura de software}

O firmware foi estruturado em três camadas principais. A camada \emph{Core}
engloba os artefatos gerados pelo STM32CubeMX, incluindo inicialização de
periféricos, descrições de pinos e funções HAL. Sobre ela, a camada
\emph{App} implementa o laço principal (\texttt{app\_poll}), o agendador de
serviços e as rotinas de inicialização específicas do projeto. A camada
\emph{Services} agrupa módulos especializados, como o gerador de passos,
o controlador PID, o serviço de homing e o roteador de mensagens SPI.

A fila de recepção SPI é mantida em memória circular, preenchida pelas
rotinas de interrupção e consumida por \texttt{app\_poll}. Respostas são
enfileiradas em \texttt{g\_app\_responses} e promovidas para o buffer de DMA
quando disponíveis. A integração com os drivers TMC5160 ocorre por meio
de uma API dedicada que abstrai comandos STEP/DIR/EN e monitora
condições de falha.

\section{Procedimentos de teste}

Os testes de validação foram conduzidos em três frentes. Primeiro,
medições com osciloscópio e analisador lógico verificaram a frequência
dos pulsos STEP e o jitter do TIM6, confirmando a estabilidade em
\SI{50}{\kilo\hertz}. Em seguida, foram realizadas varreduras de ganho
nos controladores PID para avaliar margem de fase e resposta a degraus,
utilizando logs exportados via USART1. Por fim, o pipeline SPI foi
monitorado com o cliente \texttt{cnc\_spi\_client.py}, observando a necessidade de
até três ciclos de enquete para respostas completas e avaliando o
impacto de diferentes valores de \texttt{APP\_SPI\_RESTART\_DEFER\_MAX}. Os dados
coletados subsidiam as análises apresentadas no
Capítulo~\ref{cap:resultados}.
