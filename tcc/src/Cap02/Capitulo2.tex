\chapter{Fundamentação Teórica}\label{cap:fundamentacao}

Este capítulo apresenta os conceitos fundamentais utilizados na
implementação do controlador CNC embarcado. São abordados os princípios
de sistemas CNC, a configuração de temporizadores no STM32L475, os
métodos DDA para geração de pulsos, a modelagem de motores de passo,
os controladores PID e os aspectos de comunicação SPI/USART empregados no
projeto.

\section{Sistemas CNC}

Máquinas de Controle Numérico Computadorizado interpretam instruções
codificadas (por exemplo, G-code) e convertem essas ordens em movimentos
coordenados entre múltiplos eixos, garantindo precisão repetível no
processo de usinagem ou manufatura~\cite{groover2015}. Controladores
modernos combinam processamento em tempo real com interfaces de alto
nível para planejar trajetórias, compensar erros e monitorar a execução.
A adoção de arquiteturas híbridas---com uma unidade embarcada
responsável pelo tempo real e um computador auxiliar para supervisão---
diminui a suscetibilidade a jitter e a perdas de sincronismo, mantendo a
flexibilidade de integração com sistemas de gestão.

\section{Temporizadores do STM32L475}

O microcontrolador STM32L475 disponibiliza temporizadores de uso geral e
avançado capazes de operar na faixa de \SI{80}{\mega\hertz} com alta
resolução temporal. A configuração de um temporizador baseia-se na
relação
\begin{equation}
    f_{\text{TIM}} = \frac{f_{\text{bus}}}{(PSC+1)(ARR+1)},
\end{equation}
onde $f_{\text{bus}}$ representa a frequência do barramento APB, $PSC$ o
prescaler e $ARR$ o registrador de auto-reload. O guia de aplicação
oficial descreve como esses parâmetros possibilitam gerar bases de tempo
precisas para laços de controle, captura de entradas e geração de pulsos
PWM~\cite{st_an4013}. No projeto, o TIM6 é configurado com
$PSC = 79$ e $ARR = 19$ para obter interrupções em \SI{50}{\kilo\hertz},
enquanto o TIM7 opera com $PSC = 7999$ e $ARR = 9$, produzindo um laço
de \SI{1}{\kilo\hertz}. Os temporizadores LPTIM1, TIM3 e TIM5 operam em
modo encoder para rastrear incrementalmente a posição dos eixos.

\section{Digital Differential Analyzer}

Algoritmos DDA são integradores digitais que aproximam trajetórias
contínuas por meio de incrementos discretos, amplamente utilizados em
sistemas de gráficos e em controladores de movimento para motores de
passo~\cite{fussell2003}. O método acumula um erro fracionário em cada
interação e emite um pulso quando a soma ultrapassa um limiar definido,
resultando em uma sequência de passos que aproxima a velocidade ou a
trajetória desejada. Arquiteturas clássicas de interpolação tratam o DDA
como núcleo do gerador de pulsos, responsável por alimentar os laços de
posição e velocidade que seguem a referência calculada amostra a
amostra~\cite{idc_cnc_interp,koren_reference,efficient_reference,unit3_interpolators}.
Panoramas comparativos mostram como variantes circular, linear e por
superfície mantêm avanço constante mesmo em trajetórias multi-eixo, o
que fundamenta a escolha de incrementos acumulativos sincronizados para o
firmware do STM32~\cite{koren_cnc_interpolators}. No contexto deste
trabalho, o DDA implementado no TIM6 gera sinais STEP com resolução de
\SI{20}{\micro\second} e tolera ajustes de velocidade em tempo real sem
quebrar a coesão dos múltiplos eixos. A abordagem permite sincronizar
movimentos lineares e circulares através da atualização de incrementos
acumulados por eixo durante a ISR do temporizador, preservando a
planicidade de avanço descrita pelas referências.

\section{Modelagem de motores de passo}

Motores de passo híbridos apresentam dinâmica eletromecânica dominada
por indutâncias de fase, resistência de enrolamento e um torque
relacionado à diferença angular entre rotor e campo magnético. Modelos
clássicos de motores de passo descrevem a relação entre corrente,
torque e velocidade angular por meio de equações diferenciais acopladas
que podem ser discretizadas para implementação em controladores digitais
~\cite{kenjo1994}. A precisão do controle depende da estimação do torque
de carga $T_L$, do momento de inércia equivalente $J$ e da compensação
de efeitos como ressonâncias de meia etapa. A utilização de encoders em
modo quadratura provê realimentação adicional para compensar perda de
passos e acúmulo de erro estático.

\section{Controlador PID digital}

O controlador Proporcional-Integral-Derivativo (PID) continua sendo uma
das estratégias mais difundidas para controle de processos, combinando
uma ação proporcional que reage ao erro instantâneo, um termo integral
que remove erro estacionário e um termo derivativo que prevê tendências
de variação~\cite{astrom1995}. Loops servo em máquinas CNC seguem as
posições discretizadas pelo interpolador e corrigem desvios com ganhos
sintonizados para cada eixo, podendo incluir observadores ou compensação
de distúrbios para manter a precisão em alta velocidade~\cite{efficient_reference,real_time_interpolators,kung_fpga_motion}.
Para implementação digital, é comum utilizar a forma incremental
\begin{equation}
    u[k] = u[k-1] + K_p(e[k] - e[k-1]) + K_i T_s e[k] + \frac{K_d}{T_s}(e[k] - 2e[k-1] + e[k-2]),
\end{equation}
onde $T_s$ é o período de amostragem. No laço de \SI{1}{\kilo\hertz}, o
período fixo reduz o esforço computacional e facilita a análise de
estabilidade. Estratégias derivadas, como anti-\emph{windup} e filtros de
primeira ordem no termo derivativo, são essenciais para lidar com o
ruído proveniente dos encoders e das variações do torque de
carga. Pesquisas recentes destacam ainda controladores PID
acoplados/cross-coupled que tratam o erro de contorno entre eixos como
variável adicional, reduzindo desvios em trajetórias complexas~\cite{adaptive_fuzzy_pid}.

\section{Integração PID-DDA}

A sincronização entre o DDA e o controlador PID ocorre ao transformar o
comando de posição desejada em incrementos de passos por período do TIM6.
Métodos de distribuição diferencial garantem que ajustes produzidos pelo
PID sejam refletidos sem rupturas na geração de pulsos, mantendo o
alinhamento entre eixos~\cite{mori2005dda,efficient_reference}. Materiais
didáticos e relatórios industriais ilustram o fluxo completo: o
interpolador entrega referências de posição, o encoder fornece o retorno
real e o PID calcula o esforço aplicado ao motor para cancelar o erro, em
um ciclo repetido a cada \SI{1}{\milli\second}~\cite{idc_cnc_interp,unit3_interpolators}.
Implementações modernas combinam esses blocos em FPGAs ou SoCs para
reduzir latência e habilitar interpolação multi-eixo síncrona com laços
servo dedicados~\cite{kung_fpga_motion,real_time_interpolators}. No
firmware do STM32, o laço de controle alimenta as metas de velocidade e
microstepping de cada eixo, enquanto o DDA executa as transições
discretas, possibilitando perfis suaves e respeitando os limites de
aceleração definidos pelo firmware.

\section{Comunicação SPI e USART}

A comunicação com a Raspberry Pi utiliza o periférico SPI2 em modo
escravo com DMA circular. Esse desenho reduz a carga da CPU e garante que
as transferências de 42 bytes sejam executadas dentro da janela entre
interrupções do TIM6 e TIM7. A USART1, configurada como \emph{Virtual
COM Port}, é utilizada para depuração e registro de eventos críticos,
com uma fila não bloqueante que impede o impacto sobre o laço de
controle. A coordenação entre SPI e USART é fundamental para evitar
inversões de prioridade que poderiam comprometer o determinismo do
sistema~\cite{um2153}. A análise do pipeline de polls evidencia como o
firmware pausa e reinicia o DMA para garantir que cada resposta seja
transmitida somente após processamento completo.