\section{Modelo FOPDT e síntese de ganhos PD}\label{sec:fopdt_pd}

O comportamento dinâmico da malha de velocidade foi aproximado por um
modelo de primeira ordem com atraso puro (FOPDT, do inglês
\emph{First-Order Plus Dead Time}), cuja função de transferência é dada por
\begin{equation}
    G_v(s) = \frac{K\,e^{-Ls}}{\tau s + 1},
\end{equation}
onde $K$ denota o ganho estático, $L$ o tempo morto e $\tau$ a constante
de tempo. Para um degrau de amplitude $u_0$, a resposta temporal é
\begin{equation}
    y(t) = \begin{cases}
        0, & t < L, \\
        K\,u_0\left(1 - e^{-\frac{t-L}{\tau}}\right), & t \ge L.
    \end{cases}
\end{equation}
Propriedades úteis incluem $t_{63}=L+\tau$ (63,2\% do valor de regime) e
$t_{90}\approx L+2{,}303\,\tau$.

\subsection{Derivação da resposta ao degrau}\label{subsec:fopdt_degrau}
Considere o modelo de primeira ordem com atraso puro excitado por um
degrau de amplitude $U$ atrasado de $L$ segundos, isto é,
\(u(t) = U\,H(t-L)\), onde $H(\cdot)$ é a função de Heaviside. No
domínio do tempo, o FOPDT satisfaz a equação diferencial linear
\begin{equation}
    \tau\,\dot y(t) + y(t) = K\,u(t) = K\,U\,H(t-L), \quad y(0)=0.
\end{equation}
Para $t<L$ tem-se $u(t)=0$ e, portanto, $y(t)=0$. Para $t\ge L$ define-se
\(\xi = t-L\) e resolve-se o problema deslocado
\begin{equation}
    \tau\,\frac{d}{d\xi} y(L+\xi) + y(L+\xi) = K\,U, \quad y(L^-)=0.
\end{equation}
A solução geral é soma das soluções homogênea e particular, resultando em
\begin{equation}
    y(t) = K\,U\,\bigl(1 - e^{-\frac{t-L}{\tau}}\bigr)\,H(t-L),
\end{equation}
que coincide com a expressão utilizada nas validações: para $t<L$, $y=0$;
para $t\ge L$, a saída converge exponencialmente para $K\,U$ com
constante de tempo $\tau$.

Uma dedução equivalente via transformada de Laplace parte de
\(G_v(s)=\tfrac{K\,e^{-Ls}}{\tau s + 1}\) e de um degrau de magnitude $U$,
\(U/s\). Pela propriedade de translação temporal,
\(Y(s)=G_v(s)\,\tfrac{U}{s} = \tfrac{K\,U}{s(\tau s+1)}\,e^{-Ls}\), cuja transformada
inversa fornece a mesma expressão
\(y(t)=K\,U\,(1 - e^{-(t-L)/\tau})\,H(t-L)\).

No contexto deste trabalho, a amplitude do degrau $U$ é escolhida como a
velocidade de comando em regime (denotada por $V_{\mathrm{cmd}}$),
\(U := V_{\mathrm{cmd}} := \overline v_{\mathrm{cmd}}\), estimada a
partir da média em janela final do traço de comando (vide
Seção~\ref{sec:ident_fopdt}). Essa escolha assegura que o nível de regime
previsto pelo modelo, $K\,U$, coincida com o patamar medido no encoder
quando a aproximação FOPDT é adequada.

Com base nesse modelo, adotou-se um regulador proporcional–derivativo (PD)
na malha de velocidade, mantendo $K_i=0$ por já existir ação integrativa
no nível de posição. As constantes são sintetizadas pelas regras de
Ziegler–Nichols para curva de reação (processos do tipo FOPDT), a saber
\cite{astrom1995}:
\begin{equation}
    K_p = 1{,}2\,\frac{\tau}{K\,L},\quad
    T_d = 0{,}5\,L,\quad
    K_d = K_p\,T_d,\quad
    K_i = 0.
\end{equation}
Para implementação digital, os ganhos podem ser escalonados para
representação inteira utilizando um fator de escala $S>0$ (\emph{fixed-point}),
\emph{e.g.}, $k_{p,i}=\operatorname{round}(S\,K_p)$ e análogos para $k_{i,i}$ e
$k_{d,i}$. Tal mapeamento preserva a coerência entre a análise contínua e o
controlador discreto executado em tempo real.
