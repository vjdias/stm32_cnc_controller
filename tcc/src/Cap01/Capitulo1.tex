\chapter{Introdução}\label{cap:introducao}

A popularização de máquinas de Controle Numérico Computadorizado (CNC)
depende de controladores capazes de converter instruções digitais em
movimentos sincronizados com elevado grau de previsibilidade. No
contexto de manufatura de pequeno e médio porte, soluções baseadas em
computadores pessoais apresentam custo acessível, mas sofrem com a
variabilidade de latência inerente aos sistemas operacionais de propósito
geral. Este trabalho investiga uma alternativa embarcada utilizando o
microcontrolador STM32L475, capaz de operar a \SI{80}{\mega\hertz} e
oferecer temporizadores avançados para geração de pulsos e amostragem de
dados~\cite{st_an4013}. Ao combinar a unidade embarcada com uma
Raspberry Pi responsável pela interface de alto nível, busca-se garantir
escalabilidade e facilidade de integração com pipelines de produção.

A motivação está ligada ao desenvolvimento de um controlador determinista
que gere pulsos STEP/DIR/EN em \SI{50}{\kilo\hertz}, execute o laço
PID a \SI{1}{\kilo\hertz} e mantenha comunicação confiável com o host
via SPI e USART, entregando registros de telemetria para diagnósticos. A
arquitetura proposta precisa coordenar três eixos de motores de passo
monitorados por encoders de alta resolução, sincronizar serviços de
homing e segurança, e permitir a inserção de novas rotinas sem
comprometer as janelas temporais estabelecidas.

\section{Objetivos}\label{sec:objetivos}

\subsection{Objetivo Geral}

Projetar e documentar um controlador CNC determinístico baseado no
STM32L475, integrando geração DDA de pulsos, controle PID em tempo real e
protocolo de comunicação SPI com um cliente Raspberry Pi.

\subsection{Objetivos Específicos}

\begin{itemize}
    \item Configurar o temporizador TIM6 para gerar pulsos em
          \SI{50}{\kilo\hertz} utilizando o método DDA.
    \item Implementar o loop de controle PID a \SI{1}{\kilo\hertz}
          no TIM7, incorporando leitura incremental dos encoders.
    \item Estruturar o firmware em camadas modulares (Core, App e
          Services) que facilitem a validação incremental.
    \item Validar o pipeline de comunicação SPI escravo com DMA circular
          e logs assíncronos via USART1.
    \item Registrar testes que comprovem jitter reduzido e estabilidade
          nos serviços críticos.
\end{itemize}

\section{Organização do texto}\label{sec:organizacao}

O Capítulo~\ref{cap:fundamentacao} apresenta a fundamentação teórica
sobre CNC, temporizadores STM32, DDA, controle PID e protocolos de
comunicação. O Capítulo~\ref{cap:metodologia} descreve a metodologia de
\emph{bring-up} incremental utilizada para configurar o firmware e o
hardware de suporte. Em seguida, o Capítulo~\ref{cap:resultados}
reúne os resultados experimentais, analisando desempenho dos serviços e
o comportamento do pipeline SPI. Por fim, o
Capítulo~\ref{cap:conclusao} sintetiza as contribuições, limitações e
linhas futuras de trabalho.
