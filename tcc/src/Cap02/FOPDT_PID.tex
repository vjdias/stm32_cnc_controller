\section{Modelo FOPDT e síntese de ganhos PD}\label{sec:fopdt_pd}

O comportamento dinâmico da malha de velocidade foi aproximado por um
modelo de primeira ordem com atraso puro (FOPDT, do inglês
\emph{First-Order Plus Dead Time}), cuja função de transferência é dada por
\begin{equation}
    G_v(s) = \frac{K\,e^{-Ls}}{\tau s + 1},
\end{equation}
onde $K$ denota o ganho estático, $L$ o tempo morto e $\tau$ a constante
de tempo. Para um degrau de amplitude $u_0$, a resposta temporal é
\begin{equation}
    y(t) = \begin{cases}
        0, & t < L, \\
        K\,u_0\left(1 - e^{-\frac{t-L}{\tau}}\right), & t \ge L.
    \end{cases}
\end{equation}
Propriedades úteis incluem $t_{63}=L+\tau$ (63,2\% do valor de regime) e
$t_{90}\approx L+2{,}303\,\tau$.

Com base nesse modelo, adotou-se um regulador proporcional–derivativo (PD)
na malha de velocidade, mantendo $K_i=0$ por já existir ação integrativa
no nível de posição. As constantes são sintetizadas pelas regras de
Ziegler–Nichols para curva de reação (processos do tipo FOPDT), a saber
\cite{astrom1995}:
\begin{equation}
    K_p = 1{,}2\,\frac{\tau}{K\,L},\quad
    T_d = 0{,}5\,L,\quad
    K_d = K_p\,T_d,\quad
    K_i = 0.
\end{equation}
Para implementação digital, os ganhos podem ser escalonados para
representação inteira utilizando um fator de escala $S>0$ (\emph{fixed-point}),
\emph{e.g.}, $k_{p,i}=\operatorname{round}(S\,K_p)$ e análogos para $k_{i,i}$ e
$k_{d,i}$. Tal mapeamento preserva a coerência entre a análise contínua e o
controlador discreto executado em tempo real.

