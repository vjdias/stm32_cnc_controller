\documentclass[12pt,a4paper,oneside]{easysparc}

\usepackage[T1]{fontenc}
\usepackage[utf8]{inputenc}
\usepackage{amsmath,amsfonts,mathdots,amssymb}
\usepackage{graphicx}
\usepackage{subcaption}
\usepackage{lipsum}
\usepackage{indentfirst}
\usepackage{makecell}
\usepackage{booktabs}
\usepackage{algpseudocode, algorithm}
\usepackage{hyperref}
\usepackage{pdfpages}
\usepackage{siunitx}

\DeclareMathOperator*{\argmin}{arg\,min}

\floatname{algorithm}{Algoritmo}

\onehalfspacing{}
\begin{document}
\sloppy
\author{
	Valdir Dias Silva Junior
}
\docTitle{
    Controle determinístico para CNC baseado em STM32L475 com DDA de alta frequência
}

\docType{1}

\orientador{Ricardo Menezes}
\tituloOrientador{Prof.\ Dr.}

\coorientador{}
\tituloCoorientador{}

\memberA{Ana Paula Ribeiro}
\filiationA{Profa.\ Dra., IFAL}

\memberB{Carlos Eduardo Batista}
\filiationB{MSc., SENAI CIMATEC}

\memberC{}
\filiationC{}

\memberD{}
\filiationD{}

\memberE{}
\filiationE{}

\setDate{12}
\setMonth{Novembro}
\setYear{2024}
\setLocation{Maceió, Alagoas}

\capa{}
\folhaDeRosto{}
\begin{titlepage}
    \thispagestyle{empty}
    \vspace*{6cm}
    \begin{center}
        \begin{minipage}{0.8\textwidth}
            \small
            Dados para elaboração da ficha catalográfica devem ser inseridos aqui.\newline
            Substitua este texto pelo conteúdo oficial fornecido pela biblioteca ou órgão responsável.
        \end{minipage}
    \end{center}
    \vfill
    \begin{center}
        \textit{Esta é uma página substituta gerada automaticamente para evitar anexos binários.}
    \end{center}
\end{titlepage}
\clearpage

\begin{titlepage}
    \thispagestyle{empty}
    \vspace*{5cm}
    \begin{center}
        \textbf{Ata de Aprova\c{c}\~{a}o}\\[10mm]
        Pagina destinada a ata da banca examinadora
    \end{center}
\end{titlepage}
\clearpage


\dedicatoria{}
\agradecimentos{}
\epigrafe{}

\resumo{}
\abstract{}
\listoffigures
\cleardoublepage{}
\listoftables
\cleardoublepage{}
\simbolos{}
\cleardoublepage{}
\abreviaturas{}
\cleardoublepage{}

\renewcommand{\contentsname}{Sumário}
\tableofcontents
\cleardoublepage{}

\pagenumbering{arabic}

\pagestyle{myheadings}
\renewcommand{\chaptermark}[1]{\markboth{#1}{}}
\renewcommand{\sectionmark}[1]{\markright{#1}}

\chapter{Introdução}\label{cap:introducao}

A popularização de máquinas de Controle Numérico Computadorizado (CNC)
depende de controladores capazes de converter instruções digitais em
movimentos sincronizados com elevado grau de previsibilidade. No
contexto de manufatura de pequeno e médio porte, soluções baseadas em
computadores pessoais apresentam custo acessível, mas sofrem com a
variabilidade de latência inerente aos sistemas operacionais de propósito
geral. Este trabalho investiga uma alternativa embarcada utilizando o
microcontrolador STM32L475, capaz de operar a \SI{80}{\mega\hertz} e
oferecer temporizadores avançados para geração de pulsos e amostragem de
dados~\cite{st_an4013}. Ao combinar a unidade embarcada com uma
Raspberry Pi responsável pela interface de alto nível, busca-se garantir
escalabilidade e facilidade de integração com pipelines de produção.

A motivação está ligada ao desenvolvimento de um controlador determinista
que gere pulsos STEP/DIR/EN em \SI{50}{\kilo\hertz}, execute o laço
PID a \SI{1}{\kilo\hertz} e mantenha comunicação confiável com o host
via SPI e USART, entregando registros de telemetria para diagnósticos. A
arquitetura proposta precisa coordenar três eixos de motores de passo
monitorados por encoders de alta resolução, sincronizar serviços de
homing e segurança, e permitir a inserção de novas rotinas sem
comprometer as janelas temporais estabelecidas.

\section{Objetivos}\label{sec:objetivos}

\subsection{Objetivo Geral}

Projetar e documentar um controlador CNC determinístico baseado no
STM32L475, integrando geração DDA de pulsos, controle PID em tempo real e
protocolo de comunicação SPI com um cliente Raspberry Pi.

\subsection{Objetivos Específicos}

\begin{itemize}
    \item Configurar o temporizador TIM6 para gerar pulsos em
          \SI{50}{\kilo\hertz} utilizando o método DDA.
    \item Implementar o loop de controle PID a \SI{1}{\kilo\hertz}
          no TIM7, incorporando leitura incremental dos encoders.
    \item Estruturar o firmware em camadas modulares (Core, App e
          Services) que facilitem a validação incremental.
    \item Validar o pipeline de comunicação SPI escravo com DMA circular
          e logs assíncronos via USART1.
    \item Registrar testes que comprovem jitter reduzido e estabilidade
          nos serviços críticos.
\end{itemize}

\section{Organização do texto}\label{sec:organizacao}

O Capítulo~\ref{cap:fundamentacao} apresenta a fundamentação teórica
sobre CNC, temporizadores STM32, DDA, controle PID e protocolos de
comunicação. O Capítulo~\ref{cap:metodologia} descreve a metodologia de
\emph{bring-up} incremental utilizada para configurar o firmware e o
hardware de suporte. Em seguida, o Capítulo~\ref{cap:resultados}
reúne os resultados experimentais, analisando desempenho dos serviços e
o comportamento do pipeline SPI. Por fim, o
Capítulo~\ref{cap:conclusao} sintetiza as contribuições, limitações e
linhas futuras de trabalho.

\cleardoublepage{}
\chapter{Fundamentação Teórica}\label{cap:fundamentacao}

Este capítulo apresenta os conceitos fundamentais utilizados na
implementação do controlador CNC embarcado. São abordados os princípios
de sistemas CNC, a configuração de temporizadores no STM32L475, os
métodos DDA para geração de pulsos, a modelagem de motores de passo,
os controladores PID e os aspectos de comunicação SPI/USART empregados no
projeto.

\section{Sistemas CNC}

Máquinas de Controle Numérico Computadorizado interpretam instruções
codificadas (por exemplo, G-code) e convertem essas ordens em movimentos
coordenados entre múltiplos eixos, garantindo precisão repetível no
processo de usinagem ou manufatura~\cite{groover2015}. Controladores
modernos combinam processamento em tempo real com interfaces de alto
nível para planejar trajetórias, compensar erros e monitorar a execução.
A adoção de arquiteturas híbridas---com uma unidade embarcada
responsável pelo tempo real e um computador auxiliar para supervisão---
diminui a suscetibilidade a jitter e a perdas de sincronismo, mantendo a
flexibilidade de integração com sistemas de gestão.

\section{Temporizadores do STM32L475}

O microcontrolador STM32L475 disponibiliza temporizadores de uso geral e
avançado capazes de operar na faixa de \SI{80}{\mega\hertz} com alta
resolução temporal. A configuração de um temporizador baseia-se na
relação
\begin{equation}
    f_{\text{TIM}} = \frac{f_{\text{bus}}}{(PSC+1)(ARR+1)},
\end{equation}
onde $f_{\text{bus}}$ representa a frequência do barramento APB, $PSC$ o
prescaler e $ARR$ o registrador de auto-reload. O guia de aplicação
oficial descreve como esses parâmetros possibilitam gerar bases de tempo
precisas para laços de controle, captura de entradas e geração de pulsos
PWM~\cite{st_an4013}. No projeto, o TIM6 é configurado com
$PSC = 79$ e $ARR = 19$ para obter interrupções em \SI{50}{\kilo\hertz},
enquanto o TIM7 opera com $PSC = 7999$ e $ARR = 9$, produzindo um laço
de \SI{1}{\kilo\hertz}. Os temporizadores TIM2, TIM3 e TIM5 operam em
modo encoder para rastrear incrementalmente a posição dos eixos.

\section{Digital Differential Analyzer}

Algoritmos DDA são integradores digitais que aproximam trajetórias
contínuas por meio de incrementos discretos, amplamente utilizados em
sistemas de gráficos e em controladores de movimento para motores de
passo~\cite{fussell2003}. O método acumula um erro fracionário em cada
interação e emite um pulso quando a soma ultrapassa um limiar definido,
resultando em uma sequência de passos que aproxima a velocidade ou a
trajetória desejada. Arquiteturas clássicas de interpolação tratam o DDA
como núcleo do gerador de pulsos, responsável por alimentar os laços de
posição e velocidade que seguem a referência calculada amostra a
amostra~\cite{idc_cnc_interp,koren_reference,efficient_reference,unit3_interpolators}.
Panoramas comparativos mostram como variantes circular, linear e por
superfície mantêm avanço constante mesmo em trajetórias multi-eixo, o
que fundamenta a escolha de incrementos acumulativos sincronizados para o
firmware do STM32~\cite{koren_cnc_interpolators}. No contexto deste
trabalho, o DDA implementado no TIM6 gera sinais STEP com resolução de
\SI{20}{\micro\second} e tolera ajustes de velocidade em tempo real sem
quebrar a coesão dos múltiplos eixos. A abordagem permite sincronizar
movimentos lineares e circulares através da atualização de incrementos
acumulados por eixo durante a ISR do temporizador, preservando a
planicidade de avanço descrita pelas referências.

\section{Modelagem de motores de passo}

Motores de passo híbridos apresentam dinâmica eletromecânica dominada
por indutâncias de fase, resistência de enrolamento e um torque
relacionado à diferença angular entre rotor e campo magnético. Modelos
clássicos de motores de passo descrevem a relação entre corrente,
torque e velocidade angular por meio de equações diferenciais acopladas
que podem ser discretizadas para implementação em controladores digitais
~\cite{kenjo1994}. A precisão do controle depende da estimação do torque
de carga $T_L$, do momento de inércia equivalente $J$ e da compensação
de efeitos como ressonâncias de meia etapa. A utilização de encoders em
modo quadratura provê realimentação adicional para compensar perda de
passos e acúmulo de erro estático.

\section{Controlador PID digital}

O controlador Proporcional-Integral-Derivativo (PID) continua sendo uma
das estratégias mais difundidas para controle de processos, combinando
uma ação proporcional que reage ao erro instantâneo, um termo integral
que remove erro estacionário e um termo derivativo que prevê tendências
de variação~\cite{astrom1995}. Loops servo em máquinas CNC seguem as
posições discretizadas pelo interpolador e corrigem desvios com ganhos
sintonizados para cada eixo, podendo incluir observadores ou compensação
de distúrbios para manter a precisão em alta velocidade~\cite{efficient_reference,real_time_interpolators,kung_fpga_motion}.
Para implementação digital, é comum utilizar a forma incremental
\begin{equation}
    u[k] = u[k-1] + K_p(e[k] - e[k-1]) + K_i T_s e[k] + \frac{K_d}{T_s}(e[k] - 2e[k-1] + e[k-2]),
\end{equation}
onde $T_s$ é o período de amostragem. No laço de \SI{1}{\kilo\hertz}, o
período fixo reduz o esforço computacional e facilita a análise de
estabilidade. Estratégias derivadas, como anti-\emph{windup} e filtros de
primeira ordem no termo derivativo, são essenciais para lidar com o
ruído proveniente dos encoders e das variações do torque de
carga. Pesquisas recentes destacam ainda controladores PID
acoplados/cross-coupled que tratam o erro de contorno entre eixos como
variável adicional, reduzindo desvios em trajetórias complexas~\cite{adaptive_fuzzy_pid}.

\section{Integração PID-DDA}

A sincronização entre o DDA e o controlador PID ocorre ao transformar o
comando de posição desejada em incrementos de passos por período do TIM6.
Métodos de distribuição diferencial garantem que ajustes produzidos pelo
PID sejam refletidos sem rupturas na geração de pulsos, mantendo o
alinhamento entre eixos~\cite{mori2005dda,efficient_reference}. Materiais
didáticos e relatórios industriais ilustram o fluxo completo: o
interpolador entrega referências de posição, o encoder fornece o retorno
real e o PID calcula o esforço aplicado ao motor para cancelar o erro, em
um ciclo repetido a cada \SI{1}{\milli\second}~\cite{idc_cnc_interp,unit3_interpolators}.
Implementações modernas combinam esses blocos em FPGAs ou SoCs para
reduzir latência e habilitar interpolação multi-eixo síncrona com laços
servo dedicados~\cite{kung_fpga_motion,real_time_interpolators}. No
firmware do STM32, o laço de controle alimenta as metas de velocidade e
microstepping de cada eixo, enquanto o DDA executa as transições
discretas, possibilitando perfis suaves e respeitando os limites de
aceleração definidos pelo firmware.

\section{Comunicação SPI e USART}

A comunicação com a Raspberry Pi utiliza o periférico SPI2 em modo
escravo com DMA circular. Esse desenho reduz a carga da CPU e garante que
as transferências de 42 bytes sejam executadas dentro da janela entre
interrupções do TIM6 e TIM7. A USART1, configurada como \emph{Virtual
COM Port}, é utilizada para depuração e registro de eventos críticos,
com uma fila não bloqueante que impede o impacto sobre o laço de
controle. A coordenação entre SPI e USART é fundamental para evitar
inversões de prioridade que poderiam comprometer o determinismo do
sistema~\cite{um2153}. A análise do pipeline de polls evidencia como o
firmware pausa e reinicia o DMA para garantir que cada resposta seja
transmitida somente após processamento completo.
\cleardoublepage{}
\chapter{Metodologia}\label{cap:metodologia}

A metodologia adotada para o desenvolvimento do controlador CNC seguiu
um roteiro incremental que prioriza a validação dos blocos críticos antes
de incorporar funcionalidades auxiliares. Cada etapa foi documentada,
testada e revisada de forma a garantir que os requisitos de determinismo
fossem mantidos durante todo o processo.

\section{Roteiro incremental de bring-up}

O ponto de partida consistiu na configuração do clock principal para
\SI{80}{\mega\hertz} e na definição das prioridades do NVIC de acordo
com o orçamento temporal: interrupções externas de segurança, TIM6,
SPI2/DMA, TIM7 e USART1. Em seguida, foram ativadas as entradas de parada
de emergência (E-STOP) e sensores de proximidade, assegurando que flags
de segurança pudessem interromper o fluxo de comandos. As etapas
posteriores focaram na calibração dos temporizadores: o TIM6 foi
dimensionado com $PSC = 79$ e $ARR = 19$, enquanto o TIM7 recebeu
$PSC = 7999$ e $ARR = 9$. Os temporizadores LPTIM1, TIM3 e TIM5 foram
configurados em modo quadratura para leitura de encoders.

A configuração do SPI2 em modo escravo com DMA circular foi realizada
após os temporizadores, evitando contenda na memória compartilhada.
Por fim, a USART1 foi ajustada para \SI{115200}{bps} e integrada a um
serviço de log com fila não bloqueante. Cada etapa do roteiro foi
acompanhada por testes de bancada: medições de frequência com
osciloscópio, leitura de contadores de encoder e injeção de quadros SPI
utilizando o cliente Python.

\section{Arquitetura de software}

O firmware foi estruturado em três camadas principais. A camada \emph{Core}
engloba os artefatos gerados pelo STM32CubeMX, incluindo inicialização de
periféricos, descrições de pinos e funções HAL. Sobre ela, a camada
\emph{App} implementa o laço principal (\texttt{app\_poll}), o agendador de
serviços e as rotinas de inicialização específicas do projeto. A camada
\emph{Services} agrupa módulos especializados, como o gerador de passos,
o controlador PID, o serviço de homing e o roteador de mensagens SPI.

A fila de recepção SPI é mantida em memória circular, preenchida pelas
rotinas de interrupção e consumida por \texttt{app\_poll}. Respostas são
enfileiradas em \texttt{g\_app\_responses} e promovidas para o buffer de DMA
quando disponíveis. A integração com os drivers TMC5160 ocorre por meio
de uma API dedicada que abstrai comandos STEP/DIR/EN e monitora
condições de falha.

\section{Procedimentos de teste}

Os testes de validação foram conduzidos em três frentes. Primeiro,
medições com osciloscópio e analisador lógico verificaram a frequência
dos pulsos STEP e o jitter do TIM6, confirmando a estabilidade em
\SI{50}{\kilo\hertz}. Em seguida, foram realizadas varreduras de ganho
nos controladores PID para avaliar margem de fase e resposta a degraus,
utilizando logs exportados via USART1. Por fim, o pipeline SPI foi
monitorado com o cliente \texttt{cnc\_spi\_client.py}, observando a necessidade de
até três ciclos de enquete para respostas completas e avaliando o
impacto de diferentes valores de \texttt{APP\_SPI\_RESTART\_DEFER\_MAX}. Os dados
coletados subsidiam as análises apresentadas no
Capítulo~\ref{cap:resultados}.

\cleardoublepage{}
\chapter{Resultados e Discussão}\label{cap:resultados}

Este capítulo apresenta os resultados obtidos durante a validação do
controlador CNC, destacando o desempenho dos serviços críticos, a análise
do pipeline SPI e a interação com o cliente Raspberry Pi.

\section{Desempenho dos serviços principais}

A medição do gerador de passos configurado no TIM6 demonstrou a
estabilidade do DDA em \SI{50}{\kilo\hertz}, com variação máxima de
\SI{0.4}{\percent} entre ciclos consecutivos. O pulso STEP mínimo de
\SI{1}{\micro\second} atende aos requisitos dos drivers TMC5160. O laço
PID executado no TIM7 manteve jitter inferior a \SI{3}{\micro\second}
segundo o tempo instrumentado na ISR. Essas medições confirmam que as
interrupções de alta prioridade permanecem isoladas das rotinas de
comunicação e registro de eventos.

Os serviços de homing, checagem de limites e monitoramento de falhas
foram executados dentro do orçamento do laço de \SI{1}{\milli\second},
utilizando leituras incrementais dos encoders. Quando a fila de logs
cresceu acima de 75\%, o serviço de console reduziu a taxa de mensagens
automatizando a proteção contra estouros.

\section{Análise do pipeline SPI}

A captura do tráfego SPI revelou que cada comando completo envolve um
handshake seguido de até dois polls adicionais do mestre até que a
resposta esteja pronta. Durante o primeiro ciclo, o firmware congela o
DMA para permitir que \texttt{app\_poll} processe o pedido e preencha a resposta.
Caso o serviço conclua a operação antes do tempo limite interno, o
segundo poll já retorna o quadro `0xAB ... 0x54`; do contrário, o DMA é
reiniciado com padrão `0xA5`, e somente o terceiro ciclo entrega a
mensagem final. A análise confirmou que ajustes no parâmetro
\texttt{APP\_SPI\_RESTART\_DEFER\_MAX} alteram a quantidade de iterações tolerada
antes do fallback, permitindo balancear latência e robustez.

Experimentos adicionais reduziram a cópia de memória ao promover o
payload diretamente para o buffer ativo do DMA, diminuindo o tempo médio
entre polls em \SI{18}{\percent}. Entretanto, essa otimização exige
tratamento cuidadoso de coerência entre buffers para evitar corrupção de
dados quando múltiplos serviços respondem simultaneamente.

\section{Integração com a Raspberry Pi}

O cliente \texttt{cnc\_spi\_client.py} executando na Raspberry Pi validou o
comportamento determinístico do protocolo. O script detecta automaticamente
quando a resposta não contém um frame válido e reenvia o poll após um
intervalo configurável. Durante os testes, \texttt{--tries = 4} e
\texttt{--settle-delay = 0.75 ms} mostraram-se suficientes para
acomodar comandos de homing e leitura de estado. Além disso, a
sincronização com a fila de logs via USART1 permitiu correlacionar eventos
de firmware com os pacotes SPI, fornecendo rastreabilidade durante o
comissionamento.

Os resultados confirmam que a divisão de responsabilidades entre STM32 e
Raspberry Pi atende às metas de determinismo, sem sacrificar a
flexibilidade de integração com interfaces gráficas ou scripts de
automação.

\cleardoublepage{}
\chapter{Conclusão}\label{cap:conclusao}

Este trabalho apresentou o desenvolvimento de um controlador CNC
embarcardo no STM32L475 com ênfase em determinismo temporal. A
arquitetura proposta integrou geração de pulsos DDA a
\SI{50}{\kilo\hertz}, controle PID em \SI{1}{\kilo\hertz}, leitura de
encoders em modo quadratura e comunicação SPI com DMA circular voltada à
integração com uma Raspberry Pi. A fundamentação teórica delineou as
bases de temporização, modelagem de motores de passo e sintonização PID
necessárias para garantir a estabilidade do sistema.

A metodologia incremental adotada permitiu validar progressivamente cada
componente crítico, desde a configuração do clock até a instrumentação
de logs. Os resultados indicaram jitter reduzido no gerador de passos e
no laço PID, bem como o comportamento do pipeline SPI ao lidar com polls
sequenciais do mestre. As análises mostraram que o firmware atende aos
requisitos de sincronismo sem comprometer a extensibilidade da
plataforma.

Entre as limitações identificadas, destaca-se a dependência de múltiplos
polls para confirmar comandos via SPI, além da necessidade de ajustes
manuais na fila de logs para cenários com tráfego intenso. Como trabalhos
futuros, propõe-se: (i) explorar mecanismos de reinicialização imediata
do DMA assim que um serviço disponibilizar a resposta; (ii) investigar
estratégias adaptativas de prescaler para acomodar perfis de movimento
mais agressivos; e (iii) avaliar a migração para controladores de campo
orientado (FOC) em motores de passo híbridos para reduzir vibrações em
altas velocidades.

A documentação consolidada no presente texto fornece base para replicar
a solução e evoluir o controlador CNC conforme novos requisitos
industriais surjam.

\cleardoublepage{}

\markright{Bibliografia}
\renewcommand\bibname{Bibliografia}
\bibliographystyle{apalike}
{
    \addcontentsline{toc}{chapter}{Bibliografia}
    \bibliography{Ref/SampleReferences}
}
\cleardoublepage{}

\end{document}
