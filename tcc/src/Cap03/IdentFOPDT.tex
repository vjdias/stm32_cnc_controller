\section{Identificação experimental do modelo FOPDT}\label{sec:ident_fopdt}

Esta seção descreve o procedimento de identificação adotado para
estimativa de $K$, $L$ e $\tau$ a partir de respostas a degrau de
velocidade obtidas em bancada. O método opera inteiramente no domínio do
tempo e é robusto a ruídos moderados.

\subsection{Pré-processamento e séries derivadas}\label{subsec:fopdt_preproc}
Considere amostras ordenadas por tempo contendo os campos $(t,\,p,\,c)$,
em que $p$ representa a contagem acumulada de pulsos do atuador (STEP) e
$c$ a contagem acumulada do encoder (contagens). A partir de pares ou de uma janela
deslizante de comprimento $N$, definem-se
\begin{align}
    \Delta t_k &= t_k - t_{k-N+1}, & \Delta p_k &= p_k - p_{k-N+1}, & \Delta c_k &= c_k - c_{k-N+1},\\
    v_{\mathrm{cmd}}(k) &= \frac{\Delta p_k}{\Delta t_k}, & v_{\mathrm{enc}}(k) &= \frac{\Delta c_k}{\Delta t_k}.
\end{align}
Para reduzir o efeito de ruído, utiliza-se tipicamente $N\in[5,10]$.
Pares com $\Delta t_k\le 0$ são descartados. O valor de regime (\emph{steady
state}) de cada série é a média dos últimos $\rho\in(0,1)$ da janela
temporal disponível (\emph{e.g.}, $\rho=0{,}2$), denotados por
$\overline v_{\mathrm{cmd}}$ e $\overline v_{\mathrm{enc}}$.

Nesta notação, $f(k)$ indica o valor da grandeza discreta $f$
na $k$-ésima amostra, usualmente associado ao instante $t=k\,T_s$. Em
particular, $v_{\mathrm{cmd}}(k)$ representa a velocidade de comando
(pulsos/s) derivada dos pulsos emitidos, enquanto $V_{\mathrm{cmd}}$ será
utilizado para denotar a amplitude do degrau de comando nas simulações e
validações (definido como $V_{\mathrm{cmd}} := \overline v_{\mathrm{cmd}}$).

\subsection{Extração de marcos temporais}\label{subsec:fopdt_marcos}
Define-se o \emph{início efetivo} do degrau nos sinais de comando e de
medição pelos primeiros cruzamentos de 5\% de seus respectivos regimes:
\begin{equation}
    t^{5\%}_{\mathrm{cmd}} = \inf\{ t: v_{\mathrm{cmd}}(t) \ge 0{,}05\,\overline v_{\mathrm{cmd}}\},\quad
    t^{5\%}_{\mathrm{enc}} = \inf\{ t: v_{\mathrm{enc}}(t) \ge 0{,}05\,\overline v_{\mathrm{enc}}\}.
\end{equation}
Analogamente, obtém-se o instante $t_{63}$ como o primeiro cruzamento de
63,2\% no sinal medido:
\begin{equation}
    t_{63} = \inf\{ t: v_{\mathrm{enc}}(t) \ge 0{,}632\,\overline v_{\mathrm{enc}}\}.
\end{equation}

\subsection{Estimativas $K$, $L$ e $\tau$}\label{subsec:fopdt_estimas}
O ganho estático é dado por $K=\overline v_{\mathrm{enc}}/\overline v_{\mathrm{cmd}}$.
O tempo morto é $L = \max\{0,\,t^{5\%}_{\mathrm{enc}} - t^{5\%}_{\mathrm{cmd}}\}$. A constante de
tempo decorre de $\tau = t_{63} - L$. Quando limitações de amostragem
impõem baixa resolução temporal, adota-se o piso $\Delta t_{\min}$
observado na série crua, substituindo $L$ e/ou $\tau$ por $\Delta t_{\min}$
quando as estimativas resultarem nulas ou negativas.

\subsection{Síntese PD e validação}\label{subsec:fopdt_valid}
Com as estimativas $(K, L, \tau)$, computam-se os ganhos $K_p$ e $K_d$ por
meio das regras de Ziegler–Nichols para curva de reação (com $K_i=0$ na
malha de velocidade). A validação consiste em aplicar as constantes ao
modelo FOPDT discretizado ($T_s=\SI{1}{\milli\second}$) e comparar a
resposta simulada à resposta medida, verificando sobre-elevação,
tempo de subida e ausência de saturações relevantes.

\subsection{Validação gráfica do modelo FOPDT}\label{subsec:fopdt_overlay}
Para documentar a aderência do modelo FOPDT aos dados, foram gerados
gráficos de sobreposição entre a velocidade medida (encoder) e a
resposta teórica do FOPDT a um degrau. A resposta empregada decorre da
solução analítica do sistema de 1ª ordem com atraso sob degrau atrasado
(ver Seção~\ref{subsec:fopdt_degrau}):
\(y(t) = K\,U\,[1 - e^{-(t-L)/\tau}]\) para $t\ge L$ e $0$ caso
contrário. A amplitude $U$ é tomada como a velocidade de comando em
regime, \(U := \overline v_{\mathrm{cmd}}\), garantindo que o patamar previsto
($K\,U$) coincida com o observado em regime quando a aproximação é
válida. As figuras a seguir apresentam, para cada eixo, os três
microsteppings avaliados.

\begin{figure}[H]
    \centering
    \begin{subfigure}[t]{0.32\textwidth}
        \centering
        \includegraphics[width=\linewidth]{Cap03/img/overlay_X_mstep4.png}
        \caption{Eixo X @ 1/4}
    \end{subfigure}\hfill
    \begin{subfigure}[t]{0.32\textwidth}
        \centering
        \includegraphics[width=\linewidth]{Cap03/img/overlay_X_mstep16.png}
        \caption{Eixo X @ 1/16}
    \end{subfigure}\hfill
    \begin{subfigure}[t]{0.32\textwidth}
        \centering
        \includegraphics[width=\linewidth]{Cap03/img/overlay_X_mstep256.png}
        \caption{Eixo X @ 1/256}
    \end{subfigure}
    \caption{Sobreposição entre velocidade medida e FOPDT (eixo X).}
    \label{fig:fopdt_overlay_X}
\end{figure}

\begin{figure}[H]
    \centering
    \begin{subfigure}[t]{0.32\textwidth}
        \centering
        \includegraphics[width=\linewidth]{Cap03/img/overlay_Y_mstep4.png}
        \caption{Eixo Y @ 1/4}
    \end{subfigure}\hfill
    \begin{subfigure}[t]{0.32\textwidth}
        \centering
        \includegraphics[width=\linewidth]{Cap03/img/overlay_Y_mstep16.png}
        \caption{Eixo Y @ 1/16}
    \end{subfigure}\hfill
    \begin{subfigure}[t]{0.32\textwidth}
        \centering
        \includegraphics[width=\linewidth]{Cap03/img/overlay_Y_mstep256.png}
        \caption{Eixo Y @ 1/256}
    \end{subfigure}
    \caption{Sobreposição entre velocidade medida e FOPDT (eixo Y).}
    \label{fig:fopdt_overlay_Y}
\end{figure}

\begin{figure}[H]
    \centering
    \begin{subfigure}[t]{0.32\textwidth}
        \centering
        \includegraphics[width=\linewidth]{Cap03/img/overlay_Z_mstep4.png}
        \caption{Eixo Z @ 1/4}
    \end{subfigure}\hfill
    \begin{subfigure}[t]{0.32\textwidth}
        \centering
        \includegraphics[width=\linewidth]{Cap03/img/overlay_Z_mstep16.png}
        \caption{Eixo Z @ 1/16}
    \end{subfigure}\hfill
    \begin{subfigure}[t]{0.32\textwidth}
        \centering
        \includegraphics[width=\linewidth]{Cap03/img/overlay_Z_mstep256.png}
        \caption{Eixo Z @ 1/256}
    \end{subfigure}
    \caption{Sobreposição entre velocidade medida e FOPDT (eixo Z).}
    \label{fig:fopdt_overlay_Z}
\end{figure}

Observa-se, para microsteppings altos (1/256), tempos característicos
menores ($\tau\approx\SI{1}{ms}$) e ganhos estáticos reduzidos, ao passo
que microsteppings baixos (1/4) elevam o ganho efetivo e alongam $L$ ou
$\tau$ conforme o eixo. As discrepâncias residuais decorrem de ruído de
medição, quantização e não linearidades (\emph{e.g.}, atrito viscoso e
limitações de resolução temporal), porém a forma global é adequadamente
capturada pelo FOPDT.
