\chapter{Conclusão}\label{cap:conclusao}

Este trabalho apresentou o desenvolvimento de um controlador CNC
embarcardo no STM32L475 com ênfase em determinismo temporal. A
arquitetura proposta integrou geração de pulsos DDA a
\SI{50}{\kilo\hertz}, controle PID em \SI{1}{\kilo\hertz}, leitura de
encoders em modo quadratura e comunicação SPI com DMA circular voltada à
integração com uma Raspberry Pi. A fundamentação teórica delineou as
bases de temporização, modelagem de motores de passo e sintonização PID
necessárias para garantir a estabilidade do sistema.

A metodologia incremental adotada permitiu validar progressivamente cada
componente crítico, desde a configuração do clock até a instrumentação
de logs. Os resultados indicaram jitter reduzido no gerador de passos e
no laço PID, bem como o comportamento do pipeline SPI ao lidar com polls
sequenciais do mestre. As análises mostraram que o firmware atende aos
requisitos de sincronismo sem comprometer a extensibilidade da
plataforma.

Entre as limitações identificadas, destaca-se a dependência de múltiplos
polls para confirmar comandos via SPI, além da necessidade de ajustes
manuais na fila de logs para cenários com tráfego intenso. Como trabalhos
futuros, propõe-se: (i) explorar mecanismos de reinicialização imediata
do DMA assim que um serviço disponibilizar a resposta; (ii) investigar
estratégias adaptativas de prescaler para acomodar perfis de movimento
mais agressivos; e (iii) avaliar a migração para controladores de campo
orientado (FOC) em motores de passo híbridos para reduzir vibrações em
altas velocidades.

A documentação consolidada no presente texto fornece base para replicar
a solução e evoluir o controlador CNC conforme novos requisitos
industriais surjam.
