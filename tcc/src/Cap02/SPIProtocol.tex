% Complemento da seção 2.7: Comunicação SPI — protocolo e quadros
\subsection*{Protocolo SPI e formato de quadro}

O enlace SPI escravo utiliza quadros de tamanho fixo com \textbf{42~bytes},
transferidos em modo \emph{full-duplex}. O mestre (Raspberry~Pi) envia um
\emph{poll} com a requisi\c{c}\~ao e recebe, no mesmo ciclo ou nos ciclos
subsequentes, a resposta preparada pelo firmware (via fila de respostas e
DMA). O quadro adota marcadores de in\'icio/fim para robustez:

\begin{table}[h]
  \centering
  \caption{Layout do quadro SPI (tamanho fixo: 42~bytes).}
  \label{tab:spi-frame}
  \setlength{\tabcolsep}{4pt}\footnotesize
  \begin{tabularx}{\textwidth}{lllX}
    \toprule
    Offset & Tamanho & Campo & Descri\c{c}\~ao \\
    \midrule
    0 & 1 & SOF & Marcador de in\'icio: \texttt{0xAB}. \\
    1 & 1 & CMD & Identificador do servi\c{c}o/comando. \\
    2 & 1 & FLAGS & Bits de controle. \\
    3 & 1 & LEN & Tamanho v\'alido do payload (0--34). \\
    4 & 34 & PAYLOAD & Dados do comando/resposta (preencher com \texttt{0x00} quando n\~ao usado). \\
    38 & 2 & CRC16 & CRC-16 (LSB, MSB) sobre bytes 0..37. \\
    40 & 1 & RSV & Reservado (\texttt{0x00}). \\
    41 & 1 & EOF & Marcador de fim: \texttt{0x54}. \\
    \bottomrule
  \end{tabularx}
\end{table}

Quando n\~ao h\'a resposta pronta, o escravo devolve um quadro com
\texttt{SOF=0xAB}, \texttt{CMD=0x00} e o \texttt{PAYLOAD} preenchido com
\texttt{0xA5}. Quando a resposta de um servi\c{c}o est\'a completa, o escravo
publica o quadro final com \texttt{EOF=0x54} e CRC v\'alido, conforme
observado no cliente \texttt{cnc\_spi\_client.py}.

\subsection*{Servi\c{c}o LED (exemplo de comando simples)}

O servi\c{c}o LED liga/desliga ou alterna o estado. O mestre envia uma
requisi\c{c}\~ao; a resposta confirma o estado final.

\begin{table}[h]
  \centering
  \caption{Requisi\c{c}\~ao LED (\texttt{CMD=0x10}).}
  \label{tab:spi-led-req}
  \setlength{\tabcolsep}{4pt}\footnotesize
  \begin{tabularx}{\textwidth}{lllX}
    \toprule
    Offset & Tamanho & Campo & Valor/Descri\c{c}\~ao \\
    \midrule
    0 & 1 & SOF & \texttt{0xAB} \\
    1 & 1 & CMD & \texttt{0x10} (LED) \\
    2 & 1 & FLAGS & \texttt{0x00} (padr\~ao) \\
    3 & 1 & LEN & \texttt{0x02} \\
    4 & 1 & LED\_ID & \texttt{0x00} (LED on-board) \\
    5 & 1 & MODE & \texttt{0x00=OFF}, \texttt{0x01=ON}, \texttt{0x02=TOGGLE} \\
    6..37 & 32 & RSV & \texttt{0x00} \\
    38 & 2 & CRC16 & CRC-16 sobre bytes 0..37 \\
    40 & 1 & RSV & \texttt{0x00} \\
    41 & 1 & EOF & \texttt{0x54} \\
    \bottomrule
  \end{tabularx}
\end{table}

\begin{table}[h]
  \centering
  \caption{Resposta LED.}
  \label{tab:spi-led-rsp}
  \setlength{\tabcolsep}{4pt}\footnotesize
  \begin{tabularx}{\textwidth}{lllX}
    \toprule
    Offset & Tamanho & Campo & Valor/Descri\c{c}\~ao \\
    \midrule
    0 & 1 & SOF & \texttt{0xAB} \\
    1 & 1 & CMD & \texttt{0x10} (eco) \\
    2 & 1 & FLAGS & \texttt{0x00} \\
    3 & 1 & LEN & \texttt{0x02} \\
    4 & 1 & STATUS & \texttt{0x00=OK}, \texttt{\textgreater 0}=erro \\
    5 & 1 & LED\_STATE & \texttt{0x00=OFF}, \texttt{0x01=ON} \\
    6..37 & 32 & RSV & \texttt{0x00} \\
    38 & 2 & CRC16 & CRC-16 sobre bytes 0..37 \\
    40 & 1 & RSV & \texttt{0x00} \\
    41 & 1 & EOF & \texttt{0x54} \\
    \bottomrule
  \end{tabularx}
\end{table}

\subsection*{Servi\c{c}o Movimento (exemplo de comando composto)}

O servi\c{c}o de movimento recebe metas por eixo e par\^ametros cinem\'aticos
simplificados. Campos n\~ao usados devem ser zero.

\begin{table}[h]
  \centering
  \caption{Requisi\c{c}\~ao Movimento (\texttt{CMD=0x20}).}
  \label{tab:spi-move-req}
  \setlength{\tabcolsep}{4pt}\footnotesize
  \begin{tabularx}{\textwidth}{lllX}
    \toprule
    Offset & Tamanho & Campo & Valor/Descri\c{c}\~ao \\
    \midrule
    0 & 1 & SOF & \texttt{0xAB} \\
    1 & 1 & CMD & \texttt{0x20} (MOVE) \\
    2 & 1 & FLAGS & \texttt{0x00} \\
    3 & 1 & LEN & \texttt{0x15} (exemplo) \\
    4 & 1 & AXIS\_MASK & Bit0=X, Bit1=Y, Bit2=Z \\
    5..8 & 4 & X\_STEPS & \texttt{int32} (passos relativos) \\
    9..12 & 4 & Y\_STEPS & \texttt{int32} (passos relativos) \\
    13..16 & 4 & Z\_STEPS & \texttt{int32} (passos relativos) \\
    17..20 & 4 & FEED & \texttt{uint32} (passos/s m\'ax) \\
    21 & 1 & JERK & Opcional (perfil) \\
    22..37 & 16 & RSV & \texttt{0x00} \\
    38 & 2 & CRC16 & CRC-16 sobre bytes 0..37 \\
    40 & 1 & RSV & \texttt{0x00} \\
    41 & 1 & EOF & \texttt{0x54} \\
    \bottomrule
  \end{tabularx}
\end{table}

\begin{table}[h]
  \centering
  \caption{Resposta Movimento.}
  \label{tab:spi-move-rsp}
  \setlength{\tabcolsep}{4pt}\footnotesize
  \begin{tabularx}{\textwidth}{lllX}
    \toprule
    Offset & Tamanho & Campo & Valor/Descri\c{c}\~ao \\
    \midrule
    0 & 1 & SOF & \texttt{0xAB} \\
    1 & 1 & CMD & \texttt{0x20} (eco) \\
    2 & 1 & FLAGS & \texttt{0x00} \\
    3 & 1 & LEN & \texttt{0x03} \\
    4 & 1 & STATUS & \texttt{0x00=aceito}, \texttt{0x01=ocupado}, \texttt{\textgreater 0}=erro \\
    5 & 1 & QUEUE\_DEPTH & Itens pendentes \\
    6 & 1 & RESERVED & \texttt{0x00} \\
    7..37 & 31 & RSV & \texttt{0x00} \\
    38 & 2 & CRC16 & CRC-16 sobre bytes 0..37 \\
    40 & 1 & RSV & \texttt{0x00} \\
    41 & 1 & EOF & \texttt{0x54} \\
    \bottomrule
  \end{tabularx}
\end{table}

Os exemplos acima alinham-se ao fluxo de \emph{poll} descrito na Se\c{c}\~ao~2.7.

\subsection*{SPI Raspberry Pi \texorpdfstring{$\leftrightarrow$}{<->} TMC5160: configura\c{c}\~ao por registradores}

O TMC5160 exp\'oe um barramento SPI para configura\c{c}\~ao e diagn\'ostico por meio de
transa\c{c}\~oes de \textbf{40~bits} (\,5~bytes por quadro\,), compostas por um byte de
endere\c{c}o e quatro bytes de dados. O bit mais significativo do byte de
endere\c{c}o indica leitura/escrita (\,\textbf{1}=leitura, \textbf{0}=escrita\,) e
os 7~bits menos significativos selecionam o registrador. O protocolo apresenta
\textbf{lat\^encia de uma transa\c{c}\~ao}: os dados retornados em \texttt{SDO} referem-se
ao comando emitido no quadro anterior. Recomenda-se emitir uma leitura “de
preparo” antes de coletar o valor v\'alido do registrador desejado
(ver \cite{tmc5160_ds}). Em implementa\c{c}\~oes usuais o SPI opera em modo~3
(CPOL=1, CPHA=1).

\begin{table}[h]
  \centering
  \caption{Quadro de escrita TMC5160 (5 bytes, big-endian).}
  \label{tab:tmc5160-write}
  \setlength{\tabcolsep}{4pt}\footnotesize
  \begin{tabularx}{\textwidth}{lllX}
    \toprule
    Offset & Tamanho & Campo & Descri\c{c}\~ao \\
    \midrule
    0 & 1 & ADDR[6:0], R/W=0 & Endere\c{c}o do registrador (7~bits), bit~7=0 (escrita). \\
    1..4 & 4 & DATA[31:0] & Palavra de 32~bits, ordem de bytes: MSB\,$\to$\,LSB. \\
    \bottomrule
  \end{tabularx}
\end{table}

\begin{table}[h]
  \centering
  \caption{Quadro de leitura TMC5160 (5 bytes, big-endian; resposta no ciclo seguinte).}
  \label{tab:tmc5160-read}
  \setlength{\tabcolsep}{4pt}\footnotesize
  \begin{tabularx}{\textwidth}{lllX}
    \toprule
    Offset & Tamanho & Campo & Descri\c{c}\~ao \\
    \midrule
    0 & 1 & ADDR[6:0], R/W=1 & Endere\c{c}o do registrador (7~bits), bit~7=1 (leitura). \\
    1..4 & 4 & DUMMY & \texttt{0x00}; os 32~bits lidos pertencem ao quadro anterior. \\
    \bottomrule
  \end{tabularx}
\end{table}

\noindent Exemplos de registros comuns na bring-up e opera\c{c}\~ao s\~ao apresentados junto ao TMC5160 (Se\c{c}\~ao~\ref{sec:tmc5160-examples}).
