\chapter*{Abstract}
\noindent This dissertation describes the development of a CNC controller
focused on temporal determinism, implemented on the STM32L475
microcontroller and integrated with a Raspberry Pi host. The design uses
a \SI{50}{\kilo\hertz} pulse generator based on a Digital Differential
Analyzer (DDA) running on timer TIM6, closes a \SI{1}{\kilo\hertz} PID
loop with TIM7, and leverages quadrature encoders through timers TIM2,
TIM3, and TIM5 for position estimation. Communication with the host is
handled by an SPI protocol with circular DMA, complemented by USART1
logging to preserve observability without affecting real-time
constraints.

The theoretical background covers CNC systems, stepper motor modeling,
PID control, and DDA algorithms. The methodology followed an incremental
bring-up roadmap that validated the clock tree, critical interrupts,
control loops, and communication services. Results show stable DDA output
at \SI{50}{\kilo\hertz}, PID loop jitter below \SI{3}{\micro\second}, and
the behavior of the SPI poll pipeline, highlighting the need for three
cycles to acknowledge LED commands. The work concludes by emphasizing the
robustness of the proposed firmware and suggests future optimizations for
DMA handling and auxiliary services.

\vspace{5mm}

\noindent\textbf{\textit{Keywords}:~CNC control;~STM32L475;~DDA;~PID controller;~deterministic SPI.}
