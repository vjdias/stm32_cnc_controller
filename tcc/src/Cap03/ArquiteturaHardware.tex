\section{Arquitetura de Hardware}\label{sec:arq_hw}

Esta se\c{c}\~ao descreve o circuito do controlador, os blocos f\'isicos
e as conex\~oes empregadas entre a placa STM32, os drivers de pot\^encia
TMC5160 e o encoder \'{o}ptico incremental TMCS-28. Tamb\'em s\~ao listadas
recomenda\c{c}\~oes de montagem, alimenta\c{c}\~ao e EMC.

\subsection{Vis\~ao geral do sistema}

O sistema \'e composto por:

- \textbf{L\'ogica}: placa com STM32L475 operando a \SI{3.3}{V}, cristal de
  refer\^encia e interfaces SPI/USART/GPIO.
- \textbf{Drivers de eixo}: m\'odulos/placa com \textbf{TMC5160} recebendo
  sinais \texttt{STEP}/\texttt{DIR}/\texttt{EN} e configurados via \texttt{SPI}.
- \textbf{Realimenta\c{c}\~ao}: encoder \'{o}ptico \textbf{TMCS-28} (ABN, TTL
  \SI{5}{V}) acoplado ao eixo principal.
- \textbf{Alimenta\c{c}\~ao}: trilha de \SI{5}{V} para l\'ogica e sensores;
  regulador \SI{3.3}{V} local para o microcontrolador; alimenta\c{c}\~ao separada/filtrada
  para os est\'agios de pot\^encia dos motores.

\subsection{Alimenta\c{c}\~ao e EMC}

- \textbf{\SI{5}{V} l\'ogica}: entrada regulada (fonte externa), com
  capacitores de granel (\SI{10}{\micro F} a \SI{47}{\micro F}) e
  desacoplamentos locais (\SI{100}{nF}) pr\'oximos aos CI.
- \textbf{\SI{3.3}{V}}: regulador LDO dedicado ao STM32 e perif\'ericos
  \SI{3.3}{V}. Seguir recomenda\c{c}\~oes de estabilidade do LDO (ESR dos
  capacitores) e planos de terra.
- \textbf{Aterramento}: retorno de alta corrente dos motores mantido
  separado do plano de l\'ogica; conex\~ao em estrela/pr\'oximo ao ponto de
  entrada da energia. Loops curtos para sinais comuta\c{c}\~ao.

\subsection{Drivers TMC5160}

- \textbf{Sinais}: \texttt{STEP}/\texttt{DIR}/\texttt{EN} do STM32 (\SI{3.3}{V})
  para TMC5160. Interposi\c{c}\~ao de \emph{level shifter} se necess\'ario
  conforme a placa utilizada.
- \textbf{SPI}: barramento dedicado para configura\c{c}\~ao (\texttt{SCK},
  \texttt{MOSI}, \texttt{MISO}, \texttt{CS\_x}). Resistores de termina\c{c}\~ao/
  s\'erie curtos para integridade de sinal.
- \textbf{Pot\^encia}: desacoplamento de VM com capacitores de baixa ESR;
  rota\c{c}\~ao larga para trilhas de corrente.
- \textbf{Prote\c{c}\~oes}: leitura de \texttt{DRV\_STATUS} para falhas e EN
  global intertravado pelo E-STOP.

\subsection{Encoder \'{o}ptico TMCS-28}

- \textbf{Sa\'idas}: \textbf{A}, \textbf{B} (quadratura) e \textbf{N}
  (\emph{index}) em TTL \SI{5}{V}. Adaptar n\'{\i}vel para \SI{3.3}{V}
  (divisores/\emph{level shifter}) antes de entrar no STM32.
- \textbf{Conex\~ao no STM32}: A/B em temporizadores com modo encoder
  (por exemplo, LPTIM1/TIM3/TIM5); N em entrada de captura/interrup\c{c}\~ao para
  refer\^encia de zero.
- \textbf{Resolu\c{c}\~ao}: neste TCC utiliza\-se \textbf{TMCS-28-10k}
  (625\,lpr, \textbf{40\,000\,cpr}). Convers\~ao \(\theta = 360^{\circ}\cdot
  \text{count}/40{,}000\). Velocidade: \(\text{rpm} = 60\,\text{CPS}/40{,}000\).

\subsection{Intertravamentos e seguran\c{c}a}

- \textbf{E-STOP}: linha f\'isica que desabilita \texttt{EN} dos drivers e
  gera EXTI no STM32 para parada ordenada.
- \textbf{Fins de curso}: entradas com resistores de \emph{pull-up}
  e filtros RC opcionais; priorizar roteamento distante de trilhas de
  pot\^encia.

\subsection{PCB, cabeamento e montagem}

- \textbf{Layout}: separar dom\'inios de l\'ogica e pot\^encia; manter
  pares de A/B e linhas de SPI curtos e com retorno pr\'oximo.
- \textbf{Cabeamento}: utilizar pares tr\'ancados para A/B e sinais STEP/DIR;
  blindagem quando o ambiente possuir ru\'{\i}do elevado.
- \textbf{Ordem de testes}: valida\c{c}\~ao da alimenta\c{c}\~ao (teste inicial),
  clock/USART, SPI dos TMC5160, leitura ABN do TMCS-28 e, por fim,
  acionamento de motores sem carga.

Como refer\^encia adicional de montagem e integra\c{c}\~ao de hardware, ver
\cite{romeros_tcc_ifmg_2022}.
