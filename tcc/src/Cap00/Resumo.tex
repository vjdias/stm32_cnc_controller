\chapter*{Resumo}
\noindent Este trabalho descreve o desenvolvimento de um controlador CNC
com foco em determinismo temporal, implementado sobre o microcontrolador
STM32L475 e integrado a uma Raspberry Pi. O projeto emprega o gerador de
pulsos baseado em \emph{Digital Differential Analyzer} (DDA) a
\SI{50}{\kilo\hertz} usando o temporizador TIM6, fecha o laço PID de
\SI{1}{\kilo\hertz} com o TIM7 e utiliza encoders em modo quadratura no
LPTIM1 e nos timers TIM3 e TIM5 para estimar posição. A comunicação com o host
segue um protocolo SPI com DMA circular, complementado por logs via
USART1, garantindo observabilidade sem comprometer os prazos de tempo
real.

A fundamentação teórica apresenta conceitos de CNC, modelagem de motores
de passo, controle PID e algoritmos DDA. A metodologia foi conduzida por
um roteiro incremental que validou clock, interrupções críticas, laços
de controle e serviços de comunicação. Os resultados mostram a
estabilidade do DDA em \SI{50}{\kilo\hertz}, jitter inferior a
\SI{3}{\micro\second} no laço PID e o comportamento do pipeline SPI,
destacando a necessidade de três ciclos de enquete para o \emph{ack} de
comandos de LED. O trabalho conclui evidenciando a robustez do
firmware proposto e apresenta sugestões para otimização futura do DMA e
dos serviços auxiliares.

\vspace{5mm}

\noindent\textbf{\textit{Palavras-chave}:~controle CNC;~STM32L475;~DDA;~controlador PID;~SPI determinístico.}
