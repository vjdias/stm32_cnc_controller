\chapter*{Lista de Símbolos}

\begin{description}
    \item[$f_{\text{sys}}$] \quad Frequência do clock principal do microcontrolador.
    \item[$f_{\text{TIM6}}$] \quad Frequência de interrupção do temporizador TIM6 usada pelo DDA.
    \item[$f_{\text{loop}}$] \quad Frequência do laço de controle executado no TIM7.
    \item[$N_{\text{steps}}$] \quad Número de pulsos STEP gerados para cada eixo.
    \item[$p$] \quad Contagem acumulada de pulsos do atuador (STEP).
    \item[$c$] \quad Contagem acumulada do encoder (em contagens).
    \item[$k$] \quad Índice discreto de amostragem ($k\in\mathbb{N}$), com $t = k\,T_s$.
    \item[$f(k)$] \quad Valor da grandeza discreta $f$ na $k$-ésima amostra (equivalente a $f[k]$).
    \item[$v_{\mathrm{cmd}}(k)$ / $V_{\mathrm{cmd}}$] \quad Velocidade de comando: $v_{\mathrm{cmd}}(k)$ é o valor instantâneo (pulsos STEP/s) na amostra $k$; $V_{\mathrm{cmd}}$ denota a amplitude do degrau de comando utilizada nas análises (tipicamente $V_{\mathrm{cmd}} := \overline v_{\mathrm{cmd}}$).
    \item[$\theta$] \quad Posição angular estimada a partir do encoder.
    \item[$e(t)$] \quad Erro instantâneo entre referência e posição medida.
    \item[$u(t)$] \quad Sinal de controle produzido pelo regulador PID.
    \item[$K_p, K_i, K_d$] \quad Ganhos proporcional, integral e derivativo do controlador.
    \item[$J$] \quad Momento de inércia equivalente do eixo controlado.
    \item[$T_L$] \quad Torque de carga refletido no eixo do motor de passo.
    \item[$T_s$] \quad Período de amostragem do laço de controle.
    \item[$V_{\text{bus}}$] \quad Tensão do barramento que alimenta os drivers TMC5160.
\end{description}
